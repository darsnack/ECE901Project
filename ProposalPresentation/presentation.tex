\documentclass{beamer}

\mode<presentation> {\usetheme{default}}

\usepackage{graphicx} % Allows including images
\usepackage{booktabs} % Allows the use of \toprule, \midrule and \bottomrule in tables

%----------------------------------------------------------------------------------------
%	TITLE PAGE
%----------------------------------------------------------------------------------------

\title[ECE 901 Proposal Presentation]{Comparison of CNNs on Amazon EC2 versus FPGAs} % The short title appears at the bottom of every slide, the full title is only on the title page

\author{Kyle Daruwalla and Akhil Sundararajan}
\institute[UW-Madison]{ECE 901 Fall 2016}
\date{\today}

\begin{document}

\begin{frame}
	\titlepage
\end{frame}

\begin{frame}
	\frametitle{Overview}
	\tableofcontents
\end{frame}

%----------------------------------------------------------------------------------------
%	PRESENTATION SLIDES
%----------------------------------------------------------------------------------------

\section{Introduction to CNNs and FPGAs}
\begin{frame}
	\frametitle{Convolutional Neural Networks (CNNs)}
	\begin{equation}
		\min_{f\in F}\sum_{i=1}^n \mathcal{L}(f(x_i);y_i)
		\label{eq:erm}
	\end{equation}
	\begin{itemize}
		\item Uses SGD with backpropagation to arrive at optimum
		\item Inherently serial
		\item Potential system overhead to perform per-iteration computation
	\end{itemize}
\end{frame}

\begin{frame}
	\frametitle{Field-Programmable Gate Arrays (FPGAs)}
	\begin{itemize}
		\item Reconfigurable hardware platform
		\item Common target for real-time applications
		\item Written in hardware description language (HDL)
		\item Project Catapult is targeting FPGAs for NN implementation
	\end{itemize}
\end{frame}

\section{Proposed work}
\begin{frame}
	\frametitle{Software Implementation}
	\begin{itemize}
		\item Google's TensorFlow -- describe CNNs at the layer level
		\item Amazon EC2 for deployment
		\begin{itemize}
			\item Single CPU implementation
			\item GPU implementation
		\end{itemize}
		\item {\sc{Hogwild!}} implementation for GPUs
	\end{itemize}
\end{frame}

\begin{frame}
	\frametitle{Hardware Implementation}
	\begin{itemize}
		\item Use FPGAs to build CNN structure
		\item Modularize design into filters
		\item Use controller to pass data through filters and update weights
		\item Target cost-per-iteration speedup by optimizing filter units
	\end{itemize}
\end{frame}

\section{Proposed analysis}
\begin{frame}
	\frametitle{Theoretical Analysis}
	\begin{itemize}
		\item Use FPGA implementation to define constant bounds on time per filter operation
		\item Use timing constants + CNN structure to provide theoretical bound on cost-per-iteration
		\item Analyze computational complexity in terms of this cost-per-iteration
	\end{itemize}
\end{frame}

\begin{frame}
	\frametitle{Emperical Results}
	\begin{itemize}
		\item Comparing generalization error between the CPU, GPU, Hogwild!, and FPGA implementations.
		\item Comparing convergance rates between the CPU, GPU, Hogwild!, and FPGA implementations.
		\item Provide a metric of when FPGAs might provide a larger speedup than Hogwild!
	\end{itemize}
\end{frame}

%------------------------------------------------

\begin{frame}
	\frametitle{References}
	\footnotesize{
		\begin{thebibliography}{9} % Beamer does not support BibTeX so references must be inserted manually as below
			\bibitem[Niu et al., 2011]{hogwild}
			F. Niu, et al. (2011, Nov. 11).
			\emph{Hogwild!: A Lock-Free Approach to Parallelizing Stochastic Gradient Descent} (v2) [Online].
			Available: \url{https://arxiv.org/abs/1106.5730v2}
		\end{thebibliography}
	}
\end{frame}

%------------------------------------------------

\begin{frame}
	\Huge{\centerline{The End}}
\end{frame}

\end{document} 