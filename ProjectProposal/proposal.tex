\documentclass{article}

% if you need to pass options to natbib, use, e.g.:
% \PassOptionsToPackage{numbers, compress}{natbib}
% before loading nips_2016
%
% to avoid loading the natbib package, add option nonatbib:
% \usepackage[nonatbib]{nips_2016}

\usepackage[final,nonatbib]{nips_2016}

% to compile a camera-ready version, add the [final] option, e.g.:
% \usepackage[final]{nips_2016}

\usepackage[utf8]{inputenc} % allow utf-8 input
\usepackage[T1]{fontenc}    % use 8-bit T1 fonts
\usepackage{hyperref}       % hyperlinks
\usepackage{url}            % simple URL typesetting
\usepackage{booktabs}       % professional-quality tables
\usepackage{amsfonts}       % blackboard math symbols
\usepackage{nicefrac}       % compact symbols for 1/2, etc.
\usepackage{microtype}      % microtypography
\usepackage{tikz}
\usetikzlibrary{positioning}
\usetikzlibrary{shapes.geometric}
\usetikzlibrary{calc}

\tikzset{
  multiplexer/.style={
    draw,
    trapezium,
    shape border uses incircle, 
    shape border rotate=270,
    minimum size=18pt
  }
}

\bibliographystyle{ieeetr}

\title{Project Proposal}

% The \author macro works with any number of authors. There are two
% commands used to separate the names and addresses of multiple
% authors: \And and \AND.
%
% Using \And between authors leaves it to LaTeX to determine where to
% break the lines. Using \AND forces a line break at that point. So,
% if LaTeX puts 3 of 4 authors names on the first line, and the last
% on the second line, try using \AND instead of \And before the third
% author name.

\author{
    Kyle~Daruwalla \\
    Department of Electrical and Computer Engineering \\
    University of Wisconsin -- Madison \\
    \texttt{daruwalla@wisc.edu} \\
    \And
    Akhil~Sundararajun \\
    Department of Electrical and Computer Engineering \\
    University of Wisconsin -- Madison \\
    \texttt{asundararaja@wisc.edu} \\
}

\begin{document}

\maketitle

% Each section (including abstract) has its own .tex file
% The name of the .tex file corresponds to the section title
% e.x. The subsection in the Introduction on FPGAs is called
%      intro-fpga.tex
% \input{filename} is equivalent to pasting in the raw text
% from filename.tex.
% Try to edit the filename.tex files and not proposal.tex.
% This avoids conflicts most of the time.

\begin{abstract}
    \textit{Insert abstract.}
\end{abstract}

\section{Introduction}
Machine learning systems tackle problems ranging from content filtering and recommender systems to object recognition and text classification.  In recent years, advances in deep learning have led to finding better classifiers for these problems.  In deep learning, feature extraction is performed automatically by using many layers of neural networks; each layer involves passing inputs of the previous layer through a nonlinear activation function.  Compositions of successive layers can thus correspond to learning nonlinear decision boundaries, which has contributed to successes in image classification  and speech recognition.  Training of deep neural networks is performed using the stochastic gradient descent (SGD) algorithm, but the inherently serial nature of SGD has led to exploration of potential speedups through parallelized hardware.  In this project, we seek to investigate how the computational cost per iteration of training a deep neural network depends on the hardware architecture being used. 

\subsection{Field-Programmable Gate Arrays}
Even though learning algorithms are inherently serial, speedup might be possible by using specialized hardware to reduce the cost per iteration.

Field-programmable gate arrays (FPGAs) are reconfigurable hardware chips. An FPGA is comprised of \textit{slices}, which are the fundamental hardware unit from which any designed hardware is constructed. Each slice is comprised of \textit{look-up tables} (LUTs) and \textit{flip-flops} (FFs). When reporting the resource consumption of a particular design, it is common to report the metric in terms of slices or LUTs+FFs.

Hardware on an FPGA is designed using a \textit{hardware description language} (HDL). The most common HDL is Verilog. While Verilog shares some syntax with C, it should not be confused for a sequential programming language. HDLs allow a designer to spatially describe the hardware.

FPGAs are commonly used for real-time control, because the design freedom they offer allows for lean, efficient controller design. Furthermore, designs are not hampered by hardware limitations, because the designer can create any hardware he desires. As the boundary between control theory and optimization has blurred, FPGAs have become suitable hardware platforms for machine learning algorithms such as neural networks \cite{wang2008} \cite{skodzik2013}. Similarly, FPGAs are an attractive option to make object-recognition algorithms real-time \cite{ahn2015}.

While previous work has largely focused on deployment of neural networks on FPGAs, this project will focus on the training phase. Specifically, can FPGAs be utilized to build efficient parallel hardware to speedup the lengthy training process for convolutional neural networks?

\section{Problem Definition}
Convolutional neural networks are optimized for classification of structured data, such as images, and map input data into labels through successive representation layers.  Each layer abstracts information about the layer previous through a nonlinear activation function.  
To find a classifier that abstracts patterns in input data, we would like to minimize the empirical risk on the training data, given $m$ training example-label pairs $\{(x_i,y_i)\}_{i=1}^m$.
\begin{equation}
	\min_{f\in F}\sum_{i=1}^n \mathcal{L}(f(x_i);y_i)
	\label{eq:erm}
\end{equation}
where $\mathcal{L}$ is some loss function assigns a value related to the discrepancy between the label generated by the model $f$ and the true label.  SGD is used to solve the ERM, but its serial nature motivates the investigation of methods that achieve speedup.


\section{Proposed Implementation}
In order to perform an effective comparison between software implementations of CNNs and hardware implementations, we use TensorFlow, MATLAB, and Xilinx FPGAs. TensorFlow is used to train software implementations of a given CNN structure for CIFAR-10. These CNN implementations were trained and tested on a traditional CPU. A hardware implementation of the CNN has been created for Xilinx FPGAs in Vivado. Simulations have been performed to verify its functionality and collect timing characteristics, while an equivalent MATLAB model verifies its convergence.
\subsection{TensorFlow on EC2}
\textit{Information on TensorFlow implementation on EC2. Talk about CPU baseline. Talk about speed up using GPU and Hogwild!}
\subsection{Neural Networks on FPGAs}
\textit{Information on implementing a neural network on FPGA}

\section{Proposed Analysis}
After creating an FPGA implementation of a neural network, timing characteristics for a particular CNN can be created. For example, the proposed unit-neuron will have some physical path delay, $\tau$. Using this known constant and the structure of our CNN, we would like to define a design space. This will allow a designer targeting neural networks on FPGAs to know the time per iteration as a function of $\tau$, the number of layers, and the CNN structure. This will provide some theoretical closed form for the computational complexity on an FPGA.

Additionally, after collecting empirical results, we would like to do perform the following analysis:
\begin{enumerate}
	\item Comparing generalization error between the CPU, GPU, Hogwild!, and FPGA implementations.
	\item Comparing convergance rates (in terms of seconds) for SGD with backprop between the CPU, GPU, Hogwild!, and FPGA implementations.
	\item Provide a metric of when FPGAs might provide a larger speedup than Hogwild!
\end{enumerate}

\nocite{*}
\bibliography{references}

\end{document}
