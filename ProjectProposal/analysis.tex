After creating an FPGA implementation of a neural network, timing characteristics for a particular CNN can be created. For example, the proposed unit-neuron will have some physical path delay, $\tau$. Using this known constant and the structure of our CNN, we would like to define a design space. This will allow a designer targeting neural networks on FPGAs to know the time per iteration as a function of $\tau$, the number of layers, and the CNN structure. This will provide some theoretical closed form for the computational complexity on an FPGA.

Additionally, after collecting empirical results, we would like to do perform the following analysis:
\begin{enumerate}
	\item Comparing generalization error between the CPU, GPU, Hogwild!, and FPGA implementations.
	\item Comparing convergance rates between the CPU, GPU, Hogwild!, and FPGA implementations.
	\item Provide a metric of when FPGAs might provide a larger speedup than Hogwild!
\end{enumerate}