\documentclass{beamer}

\mode<presentation> {\usetheme{default}}

\usepackage{graphicx} % Allows including images
\usepackage{booktabs} % Allows the use of \toprule, \midrule and \bottomrule in tables

%----------------------------------------------------------------------------------------
%	TITLE PAGE
%----------------------------------------------------------------------------------------

\title[ECE 901 Paper Presentation]{Convexified Convolutional Neural Networks} % The short title appears at the bottom of every slide, the full title is only on the title page

\author{Kyle Daruwalla and Akhil Sundararajan}
\institute[UW-Madison]{ECE 901 Fall 2016}
\date{\today}

\begin{document}

\begin{frame}
	\titlepage
\end{frame}

\begin{frame}
	\frametitle{Overview}
	\tableofcontents
\end{frame}

%----------------------------------------------------------------------------------------
%	PRESENTATION SLIDES
%----------------------------------------------------------------------------------------

\section{Background}
\begin{frame}
	\frametitle{Paper Overview}
	\begin{enumerate}
		\item Start generic two-layer CNN
		\item Convex relaxation
		\begin{enumerate}
			\item Linear activation -- optimize for a low-rank matrix $A$ instead of filter weights and coefficients
			\item Non-linear activation -- frame problem in terms of RKHS
		\end{enumerate}
		\item Introduce a kernel-based algorithm for CCNNs
		\item Provide theoretical guarantees on the generalization error
		\item Explain extensions like pooling and multi-layer CNNs
		\item Provide experimental results on MNIST and CIFAR-10
	\end{enumerate}
\end{frame}

\begin{frame}
	\frametitle{Convolutional Neural Networks}
	\textit{Setting up CNN problem in context of paper. (two slides)}
\end{frame}

\section{Convex Relaxations}
\subsection{Linear Activation Functions}
\begin{frame}
	\frametitle{Optimizing For Low-Rank Matrix (Linear Activation)}
	\textit{Explain the process (show figure) and emphasize optimization over nuclear norm ball}
\end{frame}

\subsection{Non-Linear Activation Functions}
\begin{frame}
	\frametitle{Reproducing-Kernel Hilbert Space (RKHS)}
	\textit{Background on RKHS and representer theorem}
\end{frame}

\begin{frame}
	\frametitle{Framing Problem Using RKHS}
	\textit{Framing problem using RKHS}
\end{frame}

\section{Algorithm}
\begin{frame}
	\frametitle{CCNN Algorithm}
	\textit{Introduce algorithm}
\end{frame}

\begin{frame}
	\frametitle{Solving ERM in CCNN Algorithm}
	\textit{Details on solving minimization problem embedded in algorithm}
\end{frame}

\section{Theoretical Results}
\begin{frame}
	\frametitle{Bounds on Generalization Error}
	\textit{Theorem 1}
\end{frame}

\begin{frame}
	\frametitle{Proof of Theorem 1}
	\textit{Proof of theorem 1}
\end{frame}

\section{Experimental Results}
\begin{frame}
	\frametitle{Experimental Results}
	\textit{State results}
\end{frame}

%------------------------------------------------

\begin{frame}
	\frametitle{References}
	\footnotesize{
		\begin{thebibliography}{9} % Beamer does not support BibTeX so references must be inserted manually as below
			\bibitem[Zhang, 2016]{zhang2016}
			Y. Zhang, et al. (2016, Sept. 4).
			\emph{Convexified Convolutional Neural Networks} (v1) [Online].
			Available: \url{https://arxiv.org/abs/1609.01000}
		\end{thebibliography}
	}
\end{frame}

%------------------------------------------------

\begin{frame}
	\Huge{\centerline{The End}}
\end{frame}

\end{document} 