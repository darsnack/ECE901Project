Even though learning algorithms are inherently serial, speedup might be possible by using specialized hardware to reduce the cost per iteration.

Field-programmable gate arrays (FPGAs) are reconfigurable hardware chips. An FPGA is comprised of \textit{slices}, which are the fundamental hardware unit from which any designed hardware is constructed. Each slice is comprised of \textit{look-up tables} (LUTs) and \textit{flip-flops} (FFs). When reporting the resource consumption of a particular design, it is common to report the metric in terms of slices or LUTs+FFs.

Hardware on an FPGA is designed using a \textit{hardware description language} (HDL). The most common HDL is Verilog. While Verilog shares some syntax with C, it should not be confused for a sequential programming language. HDLs allow a designer to spatially describe the hardware.

FPGAs are commonly used for real-time control, because the design freedom they offer allows for lean, efficient controller design. Furthermore, designs are not hampered by hardware limitations, because the designer can create any hardware he desires. As the boundary between control theory and optimization has blurred, FPGAs have become suitable hardware platforms for machine learning algorithms such as neural networks \cite{wang2008} \cite{skodzik2013}. Similarly, FPGAs are an attractive option to make object-recognition algorithms real-time \cite{ahn2015}.

While previous work has largely focused on deployment of neural networks on FPGAs, this project focuses on the training phase. Specifically, can FPGAs be utilized to build efficient parallel hardware to speedup the lengthy training process for convolutional neural networks?